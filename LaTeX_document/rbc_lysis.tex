% !TeX program = xelatex
\documentclass[12pt]{article}
\usepackage{fontspec}
\setmainfont{Times New Roman}
\usepackage{geometry}
\geometry{margin=1in}
\usepackage{amsmath, amssymb}
\usepackage{graphicx}
\usepackage{booktabs}
\usepackage{graphicx}
\usepackage{caption}
\setlength{\parindent}{1em}
\setlength{\parskip}{0.8em}
\setlength{\parindent}{0.5cm}

\title{Действие на импулсно електрично поле върху биологични мембрани}
\author{Методи Латинов}
\date{\today}

\begin{document}
	\maketitle
	
	\section{Теоретични основи}
	
	Електропорацията представлява метод за индуциране на временни или постоянни пори в биологичните мембрани чрез прилагане на импулсно електрично поле. Под въздействието на електричното поле се повишава трансмембранният потенциал, което води до нарушаване на целостта на липидния бислой и образуване на пори. В зависимост от параметрите на електричното поле (интензитет, продължителност и брой на импулсите), порите могат да бъдат обратими или необратими, като в последния случай се наблюдава клетъчен лизис.
	
	Електроиндуцираният лизис на еритроцити представлява удобен модел за изследване на ефекта на електричното поле върху клетъчните мембрани, поради липсата на ядро и относително простата структура на еритроцитната мембрана.
	
	Електропорацията се осъществява с помощта на специализиран апарат – електропоратор. Всеки електропоратор работи в определен диапазон на напрежение, като максималните стойности обикновено достигат около 1100 V. Електричното поле се прилага чрез електроди, разположени в специални кювети. Разстоянието между електродите (обикновено 2, 3 или 4 mm) определя интензитета на електричното поле и позволява обработка на клетки с различни размери.
	
	\subsection{Цел на упражнението}
	
	Целта на упражнението е да се изследва влиянието на параметрите на импулсното електрично поле – интензитет, брой и продължителност на електричните импулси – върху електроиндуцирания лизис на еритроцити, оценен чрез фотометрично измерване на оптичната плътност.
	
	\section{Подготовка за експеримента}
	
	\begin{itemize}
		\item Подготвят се 16 епендорфови епруветки, като във всяка се добавят по 1.2 ml 0.9\% разтвор на NaCl с помощта на пипета.
		
		\item Използва се предварително приготвена и промита суспензия от еритроцити. От нея се прехвърлят по 140 $\mu$l директно в кюветите с електроди, като е задължително пробата да попадне точно в канала между двата електрода. Суспензията се ресуспендира внимателно, след което кюветата се поставя в генератора на електрически импулси.
		
		\item Генераторът на импулси се настройва на 10 импулса с продължителност 0.4 ms при следните напрежения: 150 V, 200 V, 250 V, 300 V, 325 V, 350 V и 375 V.
	\end{itemize}
	
	\subsection{Контролна проба}
	
	Контролната проба не се подлага на електрично поле и служи за оценка на базовата хемолиза. Тя се приготвя чрез прехвърляне на 140 $\mu$l от еритроцитната суспензия в епендорфова епруветка, съдържаща 1.2 ml физиологичен разтвор. След многократно ресуспендиране за пълна хомогенизация, от получената суспензия се прехвърлят по 20 $\mu$l в нови епендорфови епруветки с 1.2 ml 0.9\% NaCl. По този начин се получава двукратно разредена проба. Подготвят се две повторения на контролната проба.
	
	\subsection{Експериментални проби}
	
	Следвайки описаната процедура, се подготвя първата експериментална проба, на която се прилага електрично поле с напрежение 150 V. След третирането към кюветата се добавят 140 $\mu$l физиологичен разтвор и съдържанието се ресуспендира добре. От получената суспензия се прехвърлят по 20 $\mu$l в две от предварително подготвените епендорфови епруветки с NaCl.
	
	Процедурата се повтаря за останалите проби, като при всяка следваща проба се променя напрежението на електричното поле съгласно предварително зададените стойности.
	
	\subsection{Инкубация и измерване}
	
	След обработката на всички седем проби, те се оставят за инкубация в продължение на 15 минути при стайна температура. След изтичане на инкубационния период, по 1 ml от всяка епендорфова епруветка се прехвърля в кювета за фотометрично измерване на оптичната плътност при дължина на вълната 690 nm.
	
	\vspace{1cm}
	\section{Статистическа обработка и моделиране на данните}
	
	В настоящия документ статистическата обработка на експерименталните данни се извършва изцяло чрез статистическо моделиране, като се използва байесов подход за оценка на параметрите и несигурността. Всички статистически модели са реализирани на езика \textit{Stan}, който позволява формално описание на вероятностни модели и ефективно извеждане на апостериорни разпределения чрез методи на Марковските вериги на Монте Карло (MCMC).
	
	Целият анализ, заедно с използваните данни и изходния код на моделите, е публично достъпен в следното хранилище:
	
	\begin{center}
		\texttt{https://github.com/mlatinov/University-Biomembranes-}
	\end{center}
	
	\subsection{Дефиниране на моделираните величини}
	
	Преди дефинирането на статистическия модел е необходимо ясно да се формулира обектът на моделиране. В настоящото изследване се анализира ефектът на импулсното електрично поле върху биологичните мембрани, като специално внимание се обръща на електроиндуцирания лизис на еритроцити.
	
	Степента на лизис се изразява като процент хемолиза, който представлява относителният дял на лизираните клетки в изследваната проба. Независимата променлива в анализа е интензитетът на електричното поле, изразен в волт на сантиметър (V/cm).
	
	\subsection{Интензитет на електричното поле}
	
	Интензитетът на електричното поле \(E\) се изчислява на база приложеното напрежение и разстоянието между електродите съгласно формулата:
	
	\begin{equation}
		E = \frac{V}{d},
	\end{equation}
	
	където:
	\begin{itemize}
		\item \(V\) е приложеното напрежение (V),
		\item \(d\) е разстоянието между електродите (cm).
	\end{itemize}
	
	Тази зависимост позволява директно сравнение между експериментални условия, независимо от геометрията на използваните кювети.
	
	\subsection{Определяне на процента хемолиза}
	
	Процентът на хемолиза се определя чрез фотометрично измерване на оптичната плътност на пробите и се изчислява по формулата:
	
	\begin{equation}
		\%H = \frac{OD_k - OD_n}{OD_k} \cdot 100,
	\end{equation}
	
	където:
	\begin{itemize}
		\item \(OD_k\) е оптичната плътност на контролната проба,
		\item \(OD_n\) е оптичната плътност на съответната експериментална проба.
	\end{itemize}
	
	Получената величина \(\%H\) представлява наблюдаваната степен на електроиндуциран лизис и служи като изходна променлива в статистическия модел.
	
	\subsection{Мотивация за байесовия подход}
	
	Използването на байесов статистически модел позволява:
	\begin{itemize}
		\item директна оценка на несигурността в параметрите на модела,
		\item интегриране на предварителни знания чрез избор на подходящи априорни разпределения,
		\item стабилно моделиране при ограничен брой експериментални наблюдения.
	\end{itemize}
	
	Тези предимства са особено важни в настоящото изследване, където експерименталният набор от данни е малък, а биологичната вариабилност е съществен фактор.
	
	\vspace{5cm}
	\section{Дефиниция на статистическия модел}
	
	За да дефинираме статистическия модел, е необходимо първо да формулираме вероятностния механизъм, чрез който са генерирани експерименталните данни. В настоящото изследване изходната променлива е процентът хемолиза, който по дефиниция приема стойности в интервала $(0,1)$. Това естествено мотивира третирането на наблюдаваната хемолиза като реализация на случайна величина, дефинирана върху този интервал.
	
	Подходящо вероятностно разпределение за моделиране на такива величини е бета разпределението, поради факта че неговата поддържаща област е ограничена между 0 и 1. Формално можем да запишем:
	
	\begin{equation}
		H_i \sim \text{Beta}(\alpha, \beta),
	\end{equation}
	
	където $H_i$ е наблюдаваният процент хемолиза за $i$-тата проба, а параметрите $\alpha$ и $\beta$ определят формата на разпределението.
	
	В класическата си форма параметрите $\alpha$ и $\beta$ могат да се интерпретират като броя „успехи“ и „провали“ в контекста на априорна информация. Въпреки това, тази параметризация е неудобна при регресионно моделиране, тъй като не позволява директна интерпретация на средната стойност на разпределението.
	
	Поради тази причина бета разпределението често се параметризира чрез неговата средна стойност и параметър на прецизност. В този случай моделът се записва като:
	
	\begin{equation}
		H_i \sim \text{Beta}(\mu_i \cdot \phi,\,(1 - \mu_i)\cdot \phi),
	\end{equation}
	
	където:
	\begin{itemize}
		\item $\mu_i \in (0,1)$ е очакваната стойност на процента хемолиза за $i$-тата проба,
		\item $\phi > 0$ е параметърът на прецизност, който контролира дисперсията около $\mu_i$.
	\end{itemize}
	
	В тази параметризация средната стойност на бета разпределението е:
	
	\[
	\mathbb{E}[H_i] = \mu_i,
	\]
	
	докато параметърът $\phi$ определя степента, до която наблюденията са концентрирани около тази средна стойност.
	
	\subsection{Моделиране на средната стойност чрез уравнението на Хил}
	
	След като е въведена удобна и интерпретируема параметризация на бета разпределението, следващата стъпка е дефинирането на функционалната зависимост между очакваната хемолиза $\mu_i$ и интензитета на електричното поле $E_i$.
	
	За тази цел се използва уравнението на Хил, широко прилагано при моделиране на зависимости от тип „доза–отговор“ в биологията:
	
	\begin{equation}
		\mu_i = \mu_{\min} + \frac{\mu_{\max} - \mu_{\min}}{1 + \left( \frac{K}{E_i} \right)^{n}},
	\end{equation}
	
	където параметрите имат следната интерпретация:
	
	\begin{itemize}
		\item $\mu_{\min}$ — базова (фоновa) хемолиза, отговаряща на отговор при $E \rightarrow 0$; определя долната асимптота на кривата;
		\item $\mu_{\max}$ — максимална хемолиза при много високи стойности на електричното поле ($E \rightarrow \infty$); определя горната асимптота;
		\item $K$ — стойност на електричното поле, при която очакваната хемолиза е по средата между $\mu_{\min}$ и $\mu_{\max}$ (параметър от тип ED$_{50}$);
		\item $n$ — коефициент на Хил, който контролира стръмността на прехода между ниските и високите стойности на отговора.
	\end{itemize}
	
	Уравнението на Хил позволява гъвкаво описание на нелинейната зависимост между електричното поле и процента хемолиза, като едновременно осигурява ясна биологична интерпретация на всички параметри на модела.
	
	\subsection{Задаване на априорни разпределения}
	
	Уравнението на Хил въвежда набор от параметри, които не са директно наблюдаеми и следва да бъдат оценени на база експерименталните данни. В рамките на байесовия подход това се осъществява чрез задаване на априорни разпределения за всеки параметър, които формализират предварителните ни знания и очаквания преди наблюдението на данните.
	
	Априорните разпределения играят двойна роля: от една страна те са задължителен компонент за прилагането на теоремата на Байес, а от друга — позволяват въвеждане на слабо информативни ограничения, които стабилизират модела, особено при ограничен брой наблюдения.
	
	\subsubsection{Априори за асимптотичните параметри}
	
	Параметрите $\mu_{\min}$ и $\mu_{\max}$ описват съответно минималната и максималната очаквана хемолиза и по дефиниция са ограничени в интервала $(0,1)$. Поради тази причина за тях са избрани бета априорни разпределения.
	
	Конкретно са използвани следните априори:
	
	\begin{equation}
		\mu_{\min} \sim \text{Beta}(2, 20),
		\qquad
		\mu_{\max} \sim \text{Beta}(20, 2).
	\end{equation}
	
	Тези разпределения отразяват предварителното очакване, че:
	\begin{itemize}
		\item базовата (фоновата) хемолиза $\mu_{\min}$ е ниска, но ненулева;
		\item максималната хемолиза $\mu_{\max}$ е близка до 1, без да се налага тя да бъде точно 100\%.
	\end{itemize}
	
	Използваните априори са слабо информативни и позволяват на данните да доминират апостериорната оценка, ако тя е подкрепена от наблюденията.
	
	\subsubsection{Априор за параметъра $K$}
	
	Параметърът $K$ (ED$_{50}$) представлява стойността на електричното поле, при която очакваната хемолиза е по средата между $\mu_{\min}$ и $\mu_{\max}$. Този параметър е строго положителен и поради това за него е избрано логнормално априорно разпределение:
	
	\begin{equation}
		K \sim \text{LogNormal}(\log(1500), 0.5).
	\end{equation}
	
	Този избор отразява предварителното знание, че характерният мащаб на $K$ е около $1500\,\text{V/cm}$, като същевременно допуска асиметрична вариация и по-големи отклонения към по-високи стойности, което е типично за параметри от мащабен тип.
	
	\subsubsection{Априори за параметрите на стръмност и прецизност}
	
	Коефициентът на Хил $n$ и параметърът на прецизност $\phi$ са положителни непрекъснати величини, които контролират различни аспекти на модела:
	\begin{itemize}
		\item параметърът $n$ определя стръмността на кривата „доза–отговор“;
		\item параметърът $\phi$ контролира дисперсията на наблюденията около очакваната стойност $\mu_i$.
	\end{itemize}
	
	За двата параметъра е избрано гама априорно разпределение:
	
	\begin{equation}
		n \sim \text{Gamma}(3, 1),
		\qquad
		\phi \sim \text{Gamma}(2, 0.1).
	\end{equation}
	
	Гама разпределението е естествен избор в този контекст, тъй като:
	\begin{itemize}
		\item е дефинирано върху положителната реална ос;
		\item позволява гъвкав контрол върху очакваната стойност и дисперсията;
		\item е стандартен избор за параметри, които описват интензитет, стръмност или прецизност.
	\end{itemize}
	
	По-конкретно, избраните параметри предполагат умерена стръмност на кривата и относително широка вариация в прецизността, което позволява моделът да се адаптира към реалната експериментална вариабилност.
	
	\subsection{Краен пълен модел}
	
	С цел всички компоненти на модела да бъдат представени на едно място, без да се разпръскват в различни секции, в тази подточка е показана пълната спецификация на използвания байесов модел. Тази компактна форма улеснява както възпроизводимостта на анализа, така и проследяването на връзките между отделните параметри, без необходимост от допълнителни обяснения.
	
	Пълният модел, използван последователно в целия протокол, е дефиниран както следва:
	
	\[
	H_i \sim \text{Beta}(\mu_i \cdot \phi,\,(1 - \mu_i)\cdot \phi),
	\]
	
	\[
	\mu_i = \mu_{\min} + \frac{\mu_{\max} - \mu_{\min}}{1 + \left( \frac{K}{E_i} \right)^{n}},
	\]
	
	\[
	\mu_{\min} \sim \text{Beta}(2, 20), \qquad
	\mu_{\max} \sim \text{Beta}(20, 2),
	\]
	
	\[
	K \sim \text{LogNormal}(\log(1500), 0.5), \qquad
	n \sim \text{Gamma}(3, 1), \qquad
	\phi \sim \text{Gamma}(2, 0.1).
	\]
	
	\vspace{5cm}
	\section{Резултати}
	
	Преди представянето на резултатите е необходимо да се направят няколко важни уточнения. С цел да се избегне излишно натоварване на текста, диагностичните характеристики на модела (trace plots, $\hat{R}$, ефективен размер на извадката и др.) не са включени директно в този документ. Всички диагностични проверки са налични в посоченото публично хранилище и не показват индикации за проблеми с конвергенцията или стабилността на модела.
	
	Допълнително моделът беше валидиран чрез параметрично възстановяване (posterior predictive simulation) със симулиран размер на извадката $n = 200$, което също не е представено тук, но потвърждава способността на модела да възпроизвежда данни със сходна структура.
	
	Въпреки тази валидация е важно ясно да се подчертае, че наличните експериментални данни се състоят от едва 7 наблюдения. Макар че в теоретичен смисъл байесовската статистика не изисква минимален брой наблюдения, на практика толкова малък обем данни не позволява силни емпирични заключения без допълнителни допускания или значително опростяване на модела. Поради тази причина представените резултати следва да се разглеждат като \emph{proof of concept}, а не като окончателно установени биофизични зависимости.
	
	След тези уточнения, първо ще бъдат разгледани асимптотичните параметри на модела, а след това и допълнителните величини, изчислени на тяхна основа. Всички резултати са представени под формата на постериорни разпределения — т.е. вероятностни разпределения, получени чрез байесовско обновяване — тъй като е известно на всички ,че резултатите са вероятностни разпределения ,а точките са решения.
	
	\vspace{8cm}
	 \subsection{Асимптотични параметри}
	 
	 На Фиг.~\ref{fig:posterior_hill_parameters} са представени постериорните разпределения на асимптотичните параметри от уравнението на Хил, получени чрез байесовско моделиране.
	 
	 \begin{figure}[h]
	 	\centering
	 	\includegraphics[width=0.9\textwidth]{"C:/Users/Huawei/OneDrive/Bayes_analysis/University-Biomembranes-/plots/Electroinduced_erythrocyte_lysis/hill_eq_posterior_parameters.png"}
	 	\caption{Постериони разпределения на асимптотични параметри от уравнението на Хил }
	 	\label{fig:posterior_hill_parameters}
	 \end{figure}
	 
	 Интерпретацията на тези параметри е сравнително интуитивна. Започвайки с параметъра $\mu_{\min}$, който описва базовата (спонтанна) хемолиза при липса на приложено електрично поле, очакваме неговата стойност да бъде ниска. Това очакване се потвърждава от постериорното разпределение, чиято основна маса е концентрирана около стойности приблизително до 0.1. Това предполага, че фоновата хемолиза вероятно не надвишава 10\%.
	 
	 В същото време се наблюдава ясно изразена дясна опашка (right skew), което отразява значителна несигурност относно точната стойност на $\mu_{\min}$. Тази несигурност е напълно очаквана предвид изключително ограничения брой наблюдения. Потенциален начин за редуциране на тази неопределеност би било използването на по-информативни (по-тесни) приорни разпределения, които ограничават допустимите стойности на фоновата хемолиза, но подобен подход е извън обхвата на настоящия анализ.
	 
	 Параметърът $\mu_{\max}$ описва максималната постижима хемолиза при много високи стойности на интензитета на електричното поле. Биологично е разумно да се очаква, че при достатъчно силно поле почти всички еритроцити ще лизират. В съответствие с това, основната маса на постериорното разпределение на $\mu_{\max}$ е концентрирана около стойности приблизително 0.95. Наблюдава се и лява опашка (left skew), която по-скоро отразява ограниченията на наличните данни, отколкото реалистична биологична възможност за значително по-ниски максимални стойности.
	 
	 Параметърът $K$ представлява характерната стойност на интензитета на електричното поле, при която се достига половината от максималния ефект, т.е.\ около 50\% хемолиза. Постериорното разпределение на $K$ е центрирано около стойности приблизително 1500, което е в добро съответствие с визуалните тенденции, наблюдавани в експерименталните данни. Въпреки това се забелязва наличие на високи, вероятно нереалистични стойности, което отново подчертава чувствителността на този параметър към избора на приорни разпределения и ограниченото количество данни. По-задълбочено изследване и калибриране на приорите вероятно би довело до по-добре идентифициран модел.
	 
	 Накрая, параметърът $n$ (коефициентът на Хил) е по-труден за директна интерпретация, но играе ключова роля в определянето на стръмността на прехода между ниска и висока хемолиза. В контекста на dose--response модели стойности на $n > 1$, каквито доминират в постериорното разпределение, обикновено се интерпретират като индикация за кооперативно или „switch-like“ поведение на системата.
	 
	 \subsection{Моделно-изчислени стойности}
	 
	 За да не се ограничаваме единствено до интерпретация на параметрите на модела и
	 да получим по-интуитивна представа за поведението на системата, използваме
	 постериорните разпределения на параметрите на уравнението на Хил, за да
	 изчислим допълнителни, моделно-дефинирани величини с по-директен биологичен
	 смисъл.
	 
	 \subsubsection{Динамичен диапазон на очакваната хемолиза ($\Delta\mu$)}
	 
	 Динамичният диапазон на хемолизата се дефинира като разликата между
	 минималната (базова) и максималната очаквана степен на хемолиза:
	 
	 \[
	 \Delta\mu = \mu_{\max} - \mu_{\min}.
	 \]
	 
	 Тази величина описва \emph{литичния потенциал} на еритроцитите, т.е. обхвата,
	 в който електричното поле може да индуцира допълнителен лизис над
	 спонтанния фон.
	 
	 Здравите и структурно интактни еритроцити се характеризират с ниска
	 базова хемолиза ($\mu_{\min}$ близо до нула) и висок максимален отговор
	 ($\mu_{\max}$ близо до единица), което води до голям динамичен диапазон
	 ($\Delta\mu$). Това означава, че клетките запазват мембранната си цялост
	 при слабо или липсващо електрично поле, но могат да достигнат почти пълен
	 лизис при достатъчно силна електропорация.
	 
	 Обратно, предварително увредени или нестабилни клетки биха показали
	 висока базова хемолиза и ограничен максимален отговор, което води до
	 намален динамичен диапазон и по-нисък литичен потенциал.
	 
	 \begin{figure}[h]
	 	\centering
	 	\includegraphics[width=0.9\textwidth]{C:/Users/Huawei/OneDrive/Bayes_analysis/University-Biomembranes-/plots/Electroinduced_erythrocyte_lysis/posterior_delta.png}
	 	\caption{Постериорно разпределение на динамичния диапазон на очакваната хемолиза ($\Delta\mu$).}
	 	\label{fig:posterior_dynamic_delta}
	 \end{figure}
	 
	 \vspace{1cm}
	 На Фиг.~\ref{fig:posterior_dynamic_delta} се наблюдава, че постериорното
	 разпределение на $\Delta\mu$ е концентрирано около високи стойности, което
	 свидетелства за висок литичен потенциал на изследваните еритроцити.
	 Това е в съответствие с интерпретацията на асимптотичните параметри,
	 представени на Фиг.~\ref{fig:posterior_hill_parameters}, но тук информацията
	 е обобщена в една по-интуитивна величина.
	 
	 \subsubsection{Електрично поле, необходимо за достигане на фиксирани нива на хемолиза}
	 
	 В тази секция разглеждаме модела от различна гледна точка.
	 Вместо да изследваме как степента на хемолиза зависи от приложеното електрично поле,
	 се интересуваме от обратния въпрос: \emph{каква интензивност на електричното поле е
	 	необходима, за да се постигне предварително зададено ниво на хемолиза}.
	 
	 За тази цел дефинираме набор от целеви нива на хемолиза, които са от практически интерес:
	 
	 \[
	 \mathbf{v} = \{0.10,\; 0.25,\; 0.50\},
	 \]
	 
	 където всяка стойност представлява съответно 10\%, 25\% и 50\% лизис на еритроцитите.
	 
	 Използвайки постериорните разпределения на параметрите на уравнението на Хил,
	 за всяко ниво \(v_j \in \mathbf{v}\) изчисляваме необходимото електрично поле чрез
	 обратната форма на модела:
	 
	 \[
	 E(v_j) =
	 K \left(
	 \frac{1}{\dfrac{v_j - \mu_{\min}}{\mu_{\max} - \mu_{\min}}}
	 - 1
	 \right)^{1/n}.
	 \]
	 
	 Така за всяко фиксирано ниво на хемолиза получаваме не единична стойност,
	 а постериорно разпределение на необходимото електрично поле, което отразява
	 несигурността, произтичаща от ограничените експериментални данни.
	 
	 \begin{figure}[h]
	 	\centering
	 	\includegraphics[width=0.9\textwidth]{C:/Users/Huawei/OneDrive/Bayes_analysis/University-Biomembranes-/plots/Electroinduced_erythrocyte_lysis/posterior_cr_intervals_for_h.png}
	 	\caption{Постериорни интервали на електричното поле, необходимо за достигане на фиксирани нива на хемолиза (10\%, 25\% и 50\%).}
	 	\label{fig:fixed_e}
	 \end{figure}
	 
	 На Фиг.~\ref{fig:fixed_e} се вижда ясно, че по-високите нива на хемолиза
	 изискват непропорционално по-силно електрично поле.Широчината на интервалите отразява значителната несигурност в оценките,която е очаквана предвид малкия брой експериментални наблюдения.
	 
	 \subsubsection{Стръмнина на кривата електрично поле–хемолиза}
	 За да оценим колко е стръмна кривата в диапазона на $K$ (ЕД50), използваме Hill формулата:
	 
	 \[
	 \text{slope}_{50} = \frac{(\mu_{\max} - \mu_{\min}) \cdot n}{4 \cdot K} \times 100
	 \]
	 
	 Тъй като числено $E$ е голямо, изчисляваме стръмността за $100\,E$ вместо за единица $E$. На Фиг.~\ref{fig:slope50} можем да видим проблем с модела: по-голямата част от постериорното разпределение е под 0.05 за 100 $E$. Това означава, че кривата е по-плоска от очакваното. Потенциално моделът надценява стойностите на К или подценява стойностите на n ,което е много вероятно комбиниран резултат от липсата на достатъчно данни и широки стойности на априорните разпределения за n 
	 
	 \begin{figure}[h] \centering \includegraphics[width=0.9\textwidth]{ "C:/Users/Huawei/OneDrive/Bayes_analysis/University-Biomembranes-/plots/Electroinduced_erythrocyte_lysis/posterior_hill_slope.png"} \caption{Постериорно разпределение на стръмнината електрично поле–хемолизната крива} \label{fig:slope50} 
	 \end{figure}
	 
	 Ако изчислим и условния ефект на Е от модела и го съпоставим със експерименталните стойности проблема става много по-лесен за разпознаване  Фиг.~\ref{fig:slope50}
	 но с това приключва статистическия анализ засега
	 
	 \begin{figure}[h]
	 	\centering
	 	\includegraphics[width=0.7\textwidth]{	"C:/Users/Huawei/OneDrive/Bayes_analysis/University-Biomembranes-/plots/Electroinduced_erythrocyte_lysis/conditional_effect_plot.png"}
	 	\caption{Условен ефект на полето върху хемолизира на еритроците - Сравнителна графика}
	 	\label{fig:conditional}
	 \end{figure}
	 
	 \section{Заключение}
	 
	 Настоящото изследване демонстрира възможността за интегриране на експериментални протоколи за електропорация със съвременни байесови статистически методи. Чрез прилагането на Hill модел с бета вероятностно разпределение беше възможно да се характеризира количествено зависимостта между интензитета на електричното поле и степента на електроиндуциран лизис на еритроцити.
	 
	 \subsection{Ключови резултати и интерпретация}
	 
	 \paragraph{Биологична чувствителност и литичен потенциал}
	 Получените параметри на модела позволяват да се направят няколко важни извода относно поведението на еритроцитите при електропорация:
	 
	 \begin{itemize}
	 	\item \textbf{Висок литичен потенциал}: Постериорният анализ на динамичния диапазон ($\Delta\mu = \mu_{\max} - \mu_{\min}$) показа, че еритроцитите притежават значителен резерв за лизис. Това се проявява в способността на клетките да преминат от ниска базова хемолиза (приблизително 10\%) до почти пълен лизис (приблизително 95\%) под въздействието на електричното поле. Този голям динамичен диапазон е индикация за структурна цялост на мембраната и добра физиологична състояние на изследваните клетки.
	 	
	 	\item \textbf{Характеристична чувствителност}: Параметърът $K$ (ED$_{50}$), оценен на приблизително 1500 V/cm, представлява полезна мярка за средната чувствителност на еритроцитната мембрана към електропорация. Тази стойност служи като референтна точка за сравнение с други клетъчни типове или експериментални условия.
	 	
	 	\item \textbf{Моделируеми прагови стойности}: Изчислените стойности на електричното поле, необходими за достигане на 10\%, 25\% и 50\% хемолиза, предоставят практически ориентирана информация за дизайн на експерименти. Тези прагови стойности могат да се използват за определяне на оперативни условия при биофизични експерименти или биомедицински приложения.
	 \end{itemize}
	 
	 \subsection{Методологични аспекти и ограничения}
	 
	 \paragraph{Статистическа методология}
	 Използването на байесов подход с вероятностно програмиране на \textit{Stan} доказа своята ефективност при моделиране на данни с ограничен обем. Методът позволява:
	 
	 \begin{itemize}
	 	\item \textbf{Ясна параметризация}: Формалното представяне на всички предположения чрез вероятностен модел;
	 	\item \textbf{Интегрирана оценка на несигурността}: Постериорните разпределения предоставят цялостна картина на несигурността в параметрите;
	 	\item \textbf{Възможност за включване на предварителни знания}: Априорните разпределения позволяват инкорпориране на експертни познания.
	 \end{itemize}
	 
	 \paragraph{Експериментални ограничения}
	 Важно е да се подчертаят ограниченията, произтичащи от експерименталния дизайн:
	 
	 \begin{itemize}
	 	\item \textbf{Малък брой наблюдения}: С едва 7 различни стойности на електричното поле оценките на параметрите са съпроводени със значителна статистическа несигурност. Това се проявява в ширината на постериорните разпределения и чувствителността на резултатите към избора на априорни разпределения.
	 	
	 	\item \textbf{Липса на валидационни данни}: Отсъствието на независим валидационен набор от данни ограничава възможностите за проверка на прогностичната способност на модела.
	 	
	 	\item \textbf{Опростено представяне на биологичната сложност}: Моделът на Хил, въпреки своята ефективност, представлява значително опростяване на сложните биофизични процеси, протичащи при електропорацията.
	 \end{itemize}
	 
	 \subsection{Бъдещи насоки за развитие}
	 
	 \paragraph{Експериментални разширения}
	 За по-задълбочено разбиране на процеса на електропорация се предлагат следните направления за бъдещи изследвания:
	 
	 \begin{enumerate}
	 	\item \textbf{Увеличаване на обема данни}: Включване на по-широк диапазон от стойности на електричното поле, както и изследване на влиянието на други параметри (продължителност на импулса, брой импулси, температура);
	 	
	 	\item \textbf{Сравнителни изследвания}: Прилагане на същия аналитичен подход към различни клетъчни типове или клетки при различни физиологични състояния;
	 	
	 	\item \textbf{Валидационни експерименти}: Независимо потвърждение на прогностичната способност на модела чрез нови експериментални данни.
	 \end{enumerate}
	 
	 \paragraph{Методологични подобрения}
	 От статистическа гледна точка се откриват възможности за развитие на модела:
	 
	 \begin{itemize}
	 	\item \textbf{По-сложни функционални форми}: Изследване на алтернативни уравнения за доза-отговор, които по-точно отразяват биофизичните механизми на електропорация;
	 	
	 	\item \textbf{Иерархично моделиране}: Включване на информация за индивидуалната вариабилност между експерименталните реплики;
	 	
	 	\item \textbf{Байесово сравнение на модели}: Систематично оценяване на конкурентни модели за да се избере най-адекватният;
	 	
	 	\item \textbf{Чувствителност към априори}: Систематичен анализ на влиянието на избора на априорни разпределения върху заключенията.
	 \end{itemize}
	 
	 \subsection{Обобщение и приложения}
	 
	 Настоящото изследване илюстрира как съчетанието на класически биологични експерименти със съвременни статистически методи може да обогати разбирането на сложни биофизични процеси. Въпреки ограниченията, произтичащи от малкия обем данни, разработеният подход предоставя солидна основа за бъдещи, по-мащабни изследвания.
	 
	 Получените резултати имат потенциални приложения в няколко области:
	 
	 \begin{itemize}
	 	\item \textbf{Биофизични изследвания}: Количествена характеристика на мембранната устойчивост към електропорация;
	 	
	 	\item \textbf{Биомедицински технологии}: Оптимизиране на параметрите на електропорация за доставка на лекарства или генетичен материал;
	 	
	 	\item \textbf{Образователни цели}: Илюстрация на принципите на байесово моделиране в контекста на експериментална биология;
	 	
	 	\item \textbf{Методологични разработки}: Създаване на шаблон за анализ на данни от експерименти с ограничен обем.
	 \end{itemize}
	 
	 В заключение, представеният байесов подход към моделирането на данни от експерименти по електропорация демонстрира как формалното вероятностно мислене може да се приложи за получаване на количествени оценки и измерване на несигурността в условия на ограничена информация. Този методологичен рамкwork се оказва особено ценен в началните етапи на изследване, когато експерименталните ресурси са ограничени, но е необходимо да се направят първични оценки и да се планират бъдещи, по-мащабни експерименти.
	 
	 \vspace{1cm}
	 
	 \textit{Целият анализ, включително изходния код на моделите, е публично достъпен в хранилището: \texttt{https://github.com/mlatinov/University-Biomembranes-}.}
	 
	
	 
	 


		
\end{document}















